\section{Browser IDE Vergleich}
In diesem Abschnitt werden die drei Browser IDEs
github.dev \cite{githubDevWebsite},
stackblitz.com \cite{stackblitzcomWebsite},
und
gitpod.io \cite{gitpodioWebsite}
grob miteinander vergleichen.
Dabei wird auf den Editor und die wichtigsten Features eingegangen.

\subsection{github.dev}
% TODO

\subsection{stackblitz.com}
% TODO

\subsection{gitpod.io}
% TODO


\clearpage
\section{Dokumentation der Projektumsetzung}

\subsection{Schritt-für-Schritt-Anleitung}
Zuerst wird das Basis-Repository 
\url{https://github.com/t-stefan/FHB-Assignment-Backend}
im GitHub Webinterface
\url{https://github.com/new/import} 
importiert.

% TODO: search & replace all  thomasstxyz/fhb-mcce-tmcsp-assignment-backend  with <account/repository>

\noindent
Als nächstes müssen die GitHub Actions für das Repository unter
\url{https://github.com/thomasstxyz/fhb-mcce-tmcsp-assignment-backend/settings/actions}
aktiviert werden.
Wir wählen hierfür den Punkt \verb|Allow all actions and reusable workflows| aus
und drücken anschließend auf \verb|Save|. \\

\noindent
Nun kann zu \url{https://github.com/thomasstxyz/fhb-mcce-tmcsp-assignment-backend/actions/new}
navigiert werden.
Hier wählen wir den Workflow \verb|Node.js by GitHub Actions| 
(\url{https://github.com/thomasstxyz/fhb-mcce-tmcsp-assignment-backend/new/main?filename=.github%2Fworkflows%2Fnode.js.yml&workflow_template=node.js}).
Nun öffnet sich ein beispielhafter Workflow als neue Datei im Editor.
Mit \verb|Start commit| speichern wir diesen.
Anschließend kann zu \url{https://github.com/thomasstxyz/fhb-mcce-tmcsp-assignment-backend/actions/workflows/node.js.yml}
navigiert werden und der erste Durchlauf unseres Workflows live angesehen werden. \\

\noindent
Als nächsten Schritt laden wir unser Repository in einer gitpod.io Workspace
(\url{https://gitpod.io/#https://github.com/thomasstxyz/fhb-mcce-tmcsp-assignment-backend}).
Nachdem die Workspace fertig geladen ist, initialisieren wir mit 

\begin{verbatim}
	$ npm install
\end{verbatim}

\noindent
und starten die Applikation mit 
\begin{verbatim}
	$ npm start
\end{verbatim}

\noindent
Nun öffnen wir ein zweites Terminal und installieren Mocha
\cite{mochajsorgWebsite} mit dem Befehl

\begin{verbatim}
	$ npm install mocha --save-dev
\end{verbatim}

\noindent
Nach der Installation von Mocha, öffnen wir die Datei \verb|package.json|
und ändern den Wert von \verb|scripts.test| auf \verb|"mocha --exit"|.
Wir navigieren zum ursprünglichen Terminal, und brechen den Prozess mit der
Tastenkombination \verb|Ctrl+c| ab. \\

\noindent
Anschließend hängen wir folgende Zeile an das Ende der Datei \verb|index.js|:

\begin{verbatim}
	module.exports = app;
\end{verbatim}

\noindent
Weiters installieren wir das Modul \verb|superset|.

\begin{verbatim}
	$ npm install --save-dev supertest
\end{verbatim}

\noindent
Nun legen wir ein Verzeichnis namens \verb|test| an.
In diesem Verzeichnis legen wir eine Datei \verb|index.test.js| 
mit folgendem Inhalt an:

\begin{verbatim}
const request = require('supertest');
const app = require('../index');

describe('GET /', () => {
	it('should return "Hello World!"', () => {
		return request(app)
		.get('/')
		.expect('<h1>Hello World!</h1>')
	});
	
	it('should return error 404, if /nonexistentpath is requested', () => {
		return request(app)
		.get('/non/existent/path')
		.expect(404)
	});
});

describe('POST /api/notes', () => {
	it('should return json {"error":"content missing"}, if request body is missing', () => {
		return request(app)
		.post('/api/notes')
		.expect('{"error":"content missing"}')
	});
	let query = {
		content: "my super cool note",
		important: true
	}
	it('should return 200, if json was requested with content and important fields', () => {
		return request(app)
		.post('/api/notes')
		.set('Content-type', 'application/json')
		.send( query )
		.expect(200);
	});
});

describe('GET /api/notes', () => {
	it('should return 200 and Content-Type: application/json', () => {
		return request(app)
		.get('/api/notes')
		.expect(200)
		.expect('Content-Type',/json/)
	});
});

describe('GET /api/notes/:id', () => {
	it('should return 200 and known data, if existing item with id 1 is requested', () => {
		return request(app)
		.get('/api/notes/1')
		.expect(200)
		.expect('Content-Type',/json/)
		.expect('{"id":1,"content":"HTML is easy","date":"2022-01-10T17:30:31.098Z","important":true}')
	});
});
\end{verbatim}

\noindent
Dies sind unsere Tests, welche mit dem Befehl

\begin{verbatim}
	$ npm run test
\end{verbatim}

\noindent
ausgeführt werden können. \\

\noindent
Schließlich haben wir alle nötigen Änderungen vorgenommen.
Diese müssen nur noch von gitpod zurück ins Repository gepusht werden.

\begin{verbatim}
	$ git add .
	$ git commit -m "add unit tests with mocha and superset"
	$ git push
\end{verbatim}


