\section{Einleitung}

\subsection{Aufgabenstellung}

\begin{itemize}
	\item Herausarbeiten und dokumentieren der Unterschiede zwischen der Editierung und den
	Features mit VSCode Frontend der Seiten https://github.dev/ , https://stackblitz.com/ und
	https://gitpod.io
	\item Umsetzung und Dokumentation (High Level) der Projektumsetzung (Inklusive Zugriff zum
	Source Code; entweder ein öffentliches Repo oder Einladung ins private Repo)
	\begin{itemize}
		\item Import des Projekts https://github.com/t-stefan/FHB-Assignment-Backend
	\end{itemize}
\end{itemize}

\noindent
Aus diesen Punkten ist frei zu wählen (Die Anzahl der zu wählenden Punkte ergibt sich auch der
Gruppengröße; Bei 2 Personen in der Gruppe sind mindestens 3 Punkte auszuwählen, Bei 3
Personen in der Gruppe sind mindestens 5 Punkte auszuwählen; Bei 4+ Personen sind alle Punkte zu
erledigen.): 

\begin{itemize}
	\item Anlage eines automatischen Builds der bei jedem Pull-Request in den Main läuft und
	auch bei jedem Push in den Main Branch selbst
	\item Anlage von mindestens 3 Unit Tests
	\item Aufnahme der Unit Tests in den Build für jeden Pull-Request in den Main Branch
	sowie bei jedem Push in den Main Branch selbst
	\item Installation und Konfiguration (beliebige Konfiguration von jenen die bei der
	Installation vorgeschlagen werden) von ESLint für das Projekt
	\item Aufnahme von ESLint in den Build bei jedem Pull-Request in den Main Branch
	\item Konfiguration eines Automatismus zum Update von Fremdkomponenten wenn es
	eine neue Version gibt (z.B.: snyk, Dependabot, ...)
	\item Konfiguration von Statischer Code Analyse inklusive Quality Gate(s)
	Diese sollen auch bei jedem Pull-Request in den Main Branch ausgeführt werden.
\end{itemize}

\subsection{Praktische Umsetzung}
Die beschriebenen Schritte wurden praktisch umgesetzt im GitHub Repository
\url{https://github.com/thomasstxyz/fhb-mcce-tmcsp-assignment-backend}

\clearpage

